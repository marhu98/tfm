\begin{prof}
First we prove that we prove the first statement. 
Let $\{C_k\}_{k=0}^\infty$ 
and a basis $\{\e{i}{j}\}_{i,j}$ 
with $\e{i}{j}\in C_i,\forall i$.

Let $i+1$ be the largest natural so that $C_{\geq n}$
is not in canonical form. Obviously, $i+1>0$, since $\partial | _{C_0}=0$.

Let $\e{i+1}{j}$ be the generator with smallest $j$ so that $\partial e^i_j$
is neither 0 nor another generator. Clearly, $i+j>0$, since $\partial | _{C_0}=0$.
	
Let $\partial \e{i+1}{j}=\sum \alpha_k \e{k}{j-1}$, then on the LHS, we have terms which are exact (that is, on $S_{\text{death}}$),
and those which are not. We move all the exact terms in the LHS of the form $\e{k}{j-1}=\partial \e{q}{j}$, with $q\leq i$.
$$
\partial\left(
\e{i+1}{j}
-
\sum_{q=1}^i
\e{j}{q}\alpha_{k(q)}
\right)
=
\sum \beta_k \e{k}{j-1}
$$

Here we separate two cases,
if the LHS is $0$, then we simply define 
$$
\f{j}{i+1}=
\e{i+1}{j}
-
\sum_{q=1}^i
\e{j}{q}\alpha_{k(q)}
$$

In this case, clearly the linear map $A \e{i+1}{j}=\f{i+1}{j}$ and the identity on the rest 
has an upper triangular matrix and $\partial \f{i+1}{j}=0$.

In the case we can find $\beta_{k_0}\neq 0 $ we pick such $k_0$ so it is maximal. Then:

$$
\partial
\frac{1}{\beta_{k_0}}
\left(
\e{i+1}{j}
-
\sum_{q=1}^i
\e{j}{q}\alpha_{k(q)}
\right)
=
\e{k_0}{j-1}
+
\sum_{k<k_0} \frac{\beta_k}{\beta_{k_0}} \e{k}{j-1}
$$

In this case, we define:

$$
\f{i+1}{j}
=
\frac{1}{\beta_{k_0}}
\left(
\e{i+1}{j}
-
\sum_{q=1}^i
\e{q}{j}\alpha_{k(q)}
\right);
\quad
\f{k_{0}}{j-1}
=
\e{k_0}{j-1}
+
\sum_{k<k_0} \frac{\beta_k}{\beta_{k_0}} \e{k}{j-1}
$$

Then clearly $\partial \f{i+1}{j}=\f{k_0}{j-1}$ and the obvious change of variable has the desired form.
This violates our minimality condition, therefore the first statement is proved.

Now we prove the uniqueness part: Suppose we have two bases in canonical form
$\{\e{i}{j}\}$ and $\{\f{i}{j}\}$. Then choose $i$ and $j$ so that for all $p<j$ and any $m$
or $p=j$ and $m<i$ then $\e{i}{j}=\f{i}{j}$. 

We can find coefficients $\alpha_k$ and $\beta_k$ so that:

$$
\f{i}{j}
=
\sum_{k=1}^i \alpha_k \e{k}{j},
\quad
\partial \f{i}{j} = \f{l}{j-1}
=
\sum_{n=1}^l \beta_k \e{k}{j}
$$

Hence 
$$
\partial \e{i}{j}=
\sum_{n=1}^l
\e{n}{j-1}\frac{\beta_n}{\alpha_i}
-
\sum_{k=1}^l
\partial \e{k}{j-1}\frac{\alpha_k}{\alpha_i}
$$

This proves the uniqueness of the canonical form since: $\partial \e{i}{j}=\e{m}{j-1},m>l$ and 
$\e{m}{j-1}\neq \partial a_k^j, \forall k\neq i$
 
\end{prof}

