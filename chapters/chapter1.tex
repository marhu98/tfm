\chapter{Introductory concepts}

\newcommand{\R}{\mathbb{R}}

\section{Ordered complex and canonical form}


\begin{definition}[$\R$-Filtered complex]
Let $\{C_k\}_{k=0}^{\infty}$ be a complex. 
A $\R$-filtration on $\{C_k\}_{k=0}^{\infty}$ 
is an increasing sequence of real numbers, $\{r_i\}_{i=0}^n$
so that for each $r_i$ there as associated $F_{\leq r_i}C_k\subset C_k$ for every $k$
that satisfies:

$$
\{0\}\subset
F_{\leq r_0}C_k
\subset
F_{\leq r_k}C_k
\subset
\cdots
\subset
F_{\leq r_n}C_k
=
C_k
$$
\end{definition}

As we well see, there are a lot of natural circumstances on which a filtration might arise. For example, 
the singular chain comples of a CW-complex is naturally filtered by its skeleton.

There are more structures we can put on a complex:

\begin{definition}[Complex with ordered generators]

Let $\{C_k\}_{k=0}^{\infty}$ be a chain complex, with some basis $\{e_k^i\}$.
Then we say that $\{C_k\}_{k=0}^{\infty}$ has ordered generators when we fix the order 
$e_k^i<e_l^j$ if $k<l$ or $k=l$ and $i<j$.

\end{definition}

\begin{remark}
An ordered complex is naturally filtrated.
\end{remark}


\begin{definition}[Canonical form]

Let $\{C_k\}_{k=0}^{\infty}$ be a chain complex, with some basis $\{e_k^i\}$.
Then we say that $\{C_k\}_{k=0}^{\infty}$ is in it simplest form if, $\partial e_k^i$
is either $0$ or another generator.
\end{definition}

\begin{remark}
The canonical form is equivalent to saying that we can find a basis $S$ 
of $\{C_k\}_{k=0}^{\infty}$ so that $S$
can be separated into: 
\begin{enumerate}
\item $S_H$: Generators of the homology of the complex.

\item $S_{\text{birth}}$: Births, that is, elements whose boundary is $0$, but get killed in homology by an element of higher degree.

\item $S_{\text{death}}$: Deaths, elements whose boundary is another generator. 
\end{enumerate}

That is:

$$
S=S_{\text{birth}}\sqcup
S_{\text{death}}\sqcup
 S_H
$$ 
\end{remark}

\begin{theorem}
Every chain complex with ordered generators can be reduced to 
one in canonical form by an upper-triangular change of basis.

Furthermore, the canonical form is unique.
\end{theorem}

\begin{prof}
First we prove that we prove the first statement. 
Let ${C_k}_{k=0}^\infty$ 
and a basis $\{e_i^j\}_{i,j}$ 
with $e_i^j\in C_i,\forall i$.

Let $i+1$ be the smallest natural so that $C_{\leq n}$
is not in canonical form. Obviously, $i+1>0$, since $\partial | _{C_0}=0$.

Let $e_{i+1}^j$ be the generator with smallest $j$ so that $\partial e^i_j$
is neither 0 nor another generator. Clearly, $i+j>0$, since $\partial | _{C_0}=0$.

Let $\partial e_{i+1}^j=\sum \alpha_k e_k^{j-1}$, then on the LHS, we have terms which are exact (that is, on $S_{\text{death}}$),
and those which are not. We move all the exact terms in the LHS of the form $e_{k}^{j-1}=\partial e_{q}^j$, with $q\leq i$.
$$
\partial\left(
e_{i+1}^j
-
\sum_{q=1}^i
e^j_q\alpha_{k(q)}
\right)
=
\sum \beta_k e_k^{j-1}
$$

Here we separate two cases,
if the LHS is $0$, then we simply define 
$$
f^j_{i+1}=
e_{i+1}^j
-
\sum_{q=1}^i
e^j_q\alpha_{k(q)}
$$

In this case, clearly the linear map $A e_{i+1}^j=f_{i+1}^j$ and the identity on the rest 
has an upper triangular matrix and $\partial f^j_{i+1}=0$.

In the case we can find $\beta_{k_0}\neq 0 $ we pick such $k_0$ so it is maximal. Then:

$$
\partial
\frac{1}{\beta_{k_0}}
\left(
e_{i+1}^j
-
\sum_{q=1}^i
e^j_q\alpha_{k(q)}
\right)
=
e_{k_0}^{j-1}
+
\sum_{k<k_0} \frac{\beta_k}{\beta_{k_0}} e_k^{j-1}
$$

In this case, we define:

$$
f_{i+1}^j
=
\frac{1}{\beta_{k_0}}
\left(
e_{i+1}^j
-
\sum_{q=1}^i
e^j_q\alpha_{k(q)}
\right);
\quad
f_{k_{0}}^{j-1}
=
e_{k_0}^{j-1}
+
\sum_{k<k_0} \frac{\beta_k}{\beta_{k_0}} e_k^{j-1}
$$

Then clearly $\partial f^j_{i+1}=f_{k_0}^{j-1}$ and the obvious change of variable has the desired form.
This violates our minimality condition, therefore the first statement is proved.

Now we prove the uniqueness part: Suppose we have two bases in canonical form
$\{e_i^j\}$ and $\{f_i^j\}$. Then choose $i$ and $j$ so that for all $p<j$ and any $m$
or $p=j$ and $m<i$ then $e_i^j=f_i^j$. 

We can find coefficients $\alpha_k$ and $\beta_k$ so that:

$$
f_i^j
=
\sum_{k=1}^i \alpha_k e_k^j,
\quad
\partial f_i^j = f_l^{j-1}
=
\sum_{n=1}^l \beta_k e_k^j
$$

Hence 
$$
\partial e_i^j=
\sum_{n=1}^l
e_n^{j-1}\frac{\beta_n}{\alpha_i}
-
\sum_{k=1}^l
\partial e_k^{j-1}\frac{\alpha_k}{\alpha_i}
$$

This proves the uniqueness of the canonical form since: $\partial e_i^j=e_m^{j-1},m>l$ and 
$e_m^{j-1}\neq \partial a_k^j, \forall k\neq i$
 
\end{prof}

\section{Some examples of complexes}

\begin{definition}[{\Cech} Complex]

Let $X\subset \R^n$ be a discrete subset. Then, for $d>0$, we define the \Cech
complex of level $\epsilon$ as the following simplicial set:

%$$
%X_d 
%=
%\{
%x\in\R^n:
%d(x,X)\leq d
%\} 
%$$

$$
\tilde{C}_\epsilon(X)
=
\{
\sigma \subset X :
\bigcap_{y\in\sigma} B(y,\epsilon)\neq \emptyset
\}
$$

\end{definition}

However, in practice the {\Cech} complex can get too large to handle, notice that when $\epsilon$ is large enough,
the {\Cech} complex equals the power set of $X$. One solution to this is using the 
Vietories-Rips complex.

\begin{definition}[Vietoris-Rips Complex]

\end{definition}

\section{Motivation}

%\chapter{Canonical form=Persistence diagrams}


