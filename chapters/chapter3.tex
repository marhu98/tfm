\chapter{Elder-rule Staircodes}
\label{erchapter}

In this chapter we will introduce the so-called Elder-rule staircodes, this
is technique that will allows to confidently identify $H_0(M)$ combining the information given by 
a Morse function $f$ and a metric function $d_X$.

Combining $f$ and $d_X$ in a barcode-like summary. Only that in this case we will
have a bigraduation.

The function $d_X$ need not be the euclidean distance, it could be the diffusion distance
induced by a graph and $f$ need not be the standard height function. We could also take it to be,
for example, the discrete Ricci curvature.

\section{Augmented metric spaces}



We define the $\epsilon$-Vietories-Rips complex, $VR^\sigma_\epsilon(X)$ as we did in the previous chapter. 

By taking a line $\mathcal{L}$ with strictly positive slope:
$\mathcal{L}:{ax+b,a>0}$ we can reduced the bifiltration of $VR^\sigma_\epsilon(X)$ to a single filtrated complex
by setting $F_\delta(X)=VR_\delta^{a\delta+b}(X)$

We call the barcode of this complex, $F_\delta(X)$, the k-th fibered complexk-th fibered complex barcode.

We will need another definition:

\begin{definition}[$\epsilon$-chain]
    A sequence $(x_i)_1^l$ is an $\epsilon$-chain if
    $d(x_i,x_{i+1})\leq \epsilon,\forall 1\leq\epsilon < l$


    We define $[x]_\sigma^\epsilon$ to be 
    the points of $X_\simga$ that can be reached by an $\epsilon$-chain
    starting from $x$.
\end{definition}
