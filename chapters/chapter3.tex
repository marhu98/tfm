\chapter{Elder-rule Staircodes}
\label{erchapter}

In this chapter we will introduce the so-called Elder-rule staircodes, this
is technique that will allows to confidently identify $H_0(M)$ combining the information given by 
a Morse function $f$ and a metric function $d_X$.

Combining $f$ and $d_X$ in a barcode-like summary. Only that in this case we will
have a bigraduation.

The function $d_X$ need not be the euclidean distance, it could be the diffusion distance
induced by a graph and $f$ need not be the standard height function. We could also take it to be,
for example, the discrete Ricci curvature.

\section{Staircodes}



We define the $\epsilon$-Vietories-Rips complex, $VR^\sigma_\epsilon(X)$ as we did in the previous chapter. 

By taking a line $\mathcal{L}$ with strictly positive slope:
$\mathcal{L}:{ax+b,a>0}$ we can reduced the bifiltration of $VR^\sigma_\epsilon(X)$ to a single filtrated complex
by setting $F_\delta(X)=VR_\delta^{a\delta+b}(X)$.

We call the barcode of this complex, $F_\delta(X)$, the k-th fibered complexk-th fibered complex barcode.
And we shall denote it by $M\vert_L$.

We will need another definition:

\begin{definition}[$\epsilon$-chain]
    A sequence $(x_i)_1^l$ is an $\epsilon$-chain if
    $d(x_i,x_{i+1})\leq \epsilon,\forall 1\leq\epsilon < l$


    We define $[x]_\sigma^\epsilon$ to be 
    the points of $X_\sigma$ that can be reached by an $\epsilon$-chain
    starting from $x$.
    
    Furthermore, we say that $x$ is older than $y$ if and only if
    $f(x)\leq f(x+)$. Clearly, this is an order relationship.

    Finally, we call the staircode of $x$ to be:
    $$
        I_x=\{
            \sigma,\epsilon\in\mathbb{R}\vert
            x\in X_\sigma 
            \text{ x is the oldest in } X_\sigma
        \}
    $$
    Notice this defintion might not we well defined in the case $f$ is not injective.
    In the non-injective case we simply choose an order that is compatible with the older-order. 

    The set $\mathcal{I}=\{I_x\}_{x\in X}$ the elder rule staircode,
    or ER-staircode for short.
\end{definition}


{\color{red} Insert some images here}

When we instersect a line $\mathcal{L}:{ax+b,a>0}$ with each $I_x$ we get a 
barcode, which turns out coincides with the $0$-homology of the space $X$:

\begin{theorem}\cite{cai2020}
    Let $M=H_0(VR_\cdot^\cdot(X))$ 
    and $\mathcal{I}_X$ it's ER-staircode.
    And let $\mathcal{L}$ be the line defined by:$\mathcal{L}:{ax+b,a>0}$.

    Then the barcodes of $M\vert_L$ are precisely $\{\{
        L\cap I_x\vert x\in X
    \}\}$

    Where by te use of $\{\{\}\}$ we mean we count occurences of $L\cap I_x$ with multiplicity.
\end{theorem}
