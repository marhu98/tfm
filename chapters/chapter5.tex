\chapter{Statistical Inference Using the
Morse-Smale Complex}

It is natural to think that be could treat the problem
of aproximating the homology of a manifold from a finite 
sample from a stastical point of view. 
That is what we will try to do in this chapter. It will be based on the findings of the paper: \cite{che2017}.

The main idea is to use the decomposition given in \ref{handle}.

As in the previous sections we assume we have a closed d-manifold $M^d\subset \mathbb{R}^n$
and we have a Morse function on it $f:M^d\rightarrow \mathbb{R}$.

And we call the $\omega$-limit of $x\in M$ to:
$\omega^-(x)=\displaystyle\lim_{t\to \infty} \phi^t(x)$ where $\phi^t$ is 
the flow generated by $\nabla f$.

We define the descendings manifolds as the inverse image by $\omega^-$ of $Crit_{d-k}(f)$. 
We recall this are the $k$-index critical points of $f$.
Formally:
$$
D_k=
(\omega^-)^{-1}(Crit_{d-k})
$$

Since $Crit_k$ is finite because $M$ is closed we can index it:
$Crit_k=\{c_{k,1},\ldots,c_{k,m_k}\}$ and we can define analogously:
$$
D_{k,j}=
(\omega^-)^{-1}(C_{d-k,j})
$$

Obviously, $D_k=\coprod D_{k,j}$. (By $\coprod$ we mean disjoint union). And similarly
$M=\coprod_i D_i$.

We define the ascending manifold in the same way except replacing $\phi^t$ by the flow generated by $-\nabla f$
and we denote them $A_k$.

\begin{remark}
We call $D_k$ and $A_k$
descending and ascending manifolds to mantain coherence with the article \cite{che2017}.
But in the literature it usually used a very similar concept called stable and unstable manifolds (for example
see \cite{che2014}).
And they are denoted $W^s(p,q)$ and $W^u(p,q)$.
They key difference is that stable and unstable manifolds measure the flow between two critical points, not the total
incoming (resp outcoming) flow to a critical point. Needless to say a descending (ascending) manifold is the disjoint 
union of stable (unstable) manifolds.
\end{remark}

\section{Mode clustering}


\section{Analysis of mode clustering}


\section{Some remarks}
