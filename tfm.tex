\documentclass[10pt,twoneside]{book}
%\documentclass[10pt,oneside]{book}
%\documentclass[14pt]{article}   	% use "amsart" instead of "article" for AMSLaTeX format
\pagestyle{headings}

\usepackage{natbib}
\usepackage{hyperref}

%\renewcommand\bibname{Bibliografía}
%\renewcommand\bibname{Bibliografy}

\usepackage{titlesec}

%\usepackage[spanish]{babel}
 
\titleformat{\chapter}[display]
  {\normalfont\bfseries}{}{0pt}{\Huge}

\usepackage{blindtext}
\usepackage[T1]{fontenc}
%\usepackage[utf8]{inputenc}

\usepackage{bbm}

\usepackage[margin=1in,footskip=.25in]{geometry}
%\usepackage[margin=1in]{geometry}
\usepackage{geometry}
\usepackage{mathtools}    
\usepackage[latin9]{inputenc}  	
\usepackage[bottom]{footmisc}	% See geometry.pdf to learn the layout options. There are lots.
\geometry{letterpaper}                   		
\usepackage{graphicx}														
\usepackage{amssymb}
\usepackage{ragged2e}
\newcommand{\kfour}{\mathbb{K}^4}
\newcommand{\ktwo}{\mathbb{K}^2}
\newcommand{\pthree}{\mathbb{P}^3}
\newcommand{\afin}{\mathbb{A}}
\newcommand{\ov}{\overrightarrow}


\usepackage{tikz-cd}


\usepackage{faktor}


\usepackage{mathtools}


%\usepackage[spanish]{babel} % español
%\usepackage[utf8]{inputenc} % acentos sin codigo
%\usepackage{graphicx} % graficos
%\usepackage{amssymb} 
%\usepackage{amsmath,amsthm} 
%\usepackage{amsthm} 

%\usepackage{hyperref}
%\usepackage{mathtools}
\usepackage{mathrsfs} % Letras caligráficas
\usepackage{bm}



\usepackage{xcolor}

\newcommand{\mcm}{\qopname \relax o{mcm}}

\newcommand{\massey}[3]{\langle[ {#1} ],[ {#2} ],[ {#3} ]\rangle}
\newcommand{\module}[1]{\vert #1 \lvert }
\newcommand{\dotprod}[2]{\langle #1,#2 \rangle}
\newcommand{\mbot}{{(\bot)}}
%\newcommand{\isoequal}{\stackrel{\text{iso}}{=}}
\newcommand{\isoequal}{\simeq}
\newcommand{\difequal}{\stackrel{\text{dif}}{=}}
\newcommand{\defequal}{\stackrel{\text{def}}{=}}
\newcommand{\isoapprox}{\stackrel{\text{iso}}{\displaystyle\approx}}
\newcommand{\Cech}{\v{C}ech}
\newcommand{\difpartial}[2]{\frac{\partial #1}{\partial #2}}
\newcommand{\difantipartial}[2]{\frac{\partial #1}{\overline{\partial} #2}}
\newcommand{\antipartial}{{\overline{\partial }}}

\newcommand{\comilla}{{}"{}}


\newcommand{\e}[2]{e_{#1}^{(#2)}}

\newcommand{\f}[2]{f_{#1}^{(#2)}}


\usepackage{amsthm}

\newcounter{fakecnt}[section]
\def\thefakecnt{\arabic{section}}

\newtheorem{theorem}{Theorem}
\newtheorem{proposition}{Proposition}
\newtheorem*{conjecture}{Conjecture}
\newtheorem*{remark}{Remark}
\newtheorem{corollary}{Corollary}[theorem]
\newtheorem{lemma}{Lemma}[chapter]
\newtheorem{observation}{Observation}[chapter]
\newtheorem{example}{Example}[chapter]
%\renewcommand{proof}{\prof}
\theoremstyle{definition}
	\newtheorem{definition}{Definition}[chapter]
	

\newenvironment{hproof}{%
  \renewcommand{\proofname}{Esquema de demostraci�n}\proof}{\endproof}
  
 \newenvironment{prof}{%
  \renewcommand{\proofname}{Proof}\proof}{\endproof}



\renewcommand\qedsymbol{$QED$}


%\footskip = 0pt

\graphicspath{ {imgs/} }

\date{}
\begin{document}
%\autor{Mario Marhuenda Beltrán}

%\tableofcontents

\newcommand{\R}{\mathbb{R}}
\chapter{Morse theory: A quick review}
\begin{notation}

We will consider all complexes to be over a field
$\mathbb{K}$ that will either be 
 $\mathbb{Q}$,
 $\mathbb{R}$,
 or 
  $\mathbb{C}$.
Unless otherwise stated.
\end{notation}


\begin{definition}[Differential complex]
We say that a collection of indexed abelian groups
$\{C_i\}_{i=0}^{\infty}$ is a differential complex if
it has a linear operator:
$\delta_i: C_{i+1}\to C_i$
so that $\delta_i\circ\delta_{i+1}=0$.

Usually, we shall consider $\delta: C\to C$ where $C=\oplus C_i$
and $\delta$ is defined as the unique linear operator that coincides with $\delta_i$
when the domain is restricted to $C_i$.

And so, we may write that $\{C_i\}_{i=0}^{\infty}$ 
is a differential complex if and only if $\delta^2=0$.
\end{definition}

Given a manifold we can build a nice function on it, $f$,
so that it's sublevels $M^t=f^{-1}(-\infty,t]$ give us a natural way to 
deduce a CW-structure for the manifold.

In fact, these same function give us in a natural way the Morse-Smale complex,
which allows us to compute the homology of the manifold.

\begin{definition}[Morse function]

Let $M$ be a manifold, and $f\in C^\infty(M,\mathbb{R})$.
We say that $f$ is a morse function if the set of critical points
of $f$ is isolated. 

That is, if:
$
Crit(f)=\{x\in M: df\vert_x=0
\}
$
is isolated.

\end{definition}

We say that a critical point of $f$ has index $k$,
if the hessian of $f$ has index $k$ at the critical point.

We denote the critical points of index $k$ as $Crit_k(f)$.

It is well known that for any given manifold there are plenty of Morse functions,
in particular we have the following:


\begin{theorem}
\cite{mat1997}
Let $g\in C^\infty(M,\mathbb{R})$, and let $\epsilon>0$
then there exists $f\in C^\infty(M,\mathbb{R})$ so that $f$
is a Morse function and $f$ and $g$ are $C^2$-close.
\end{theorem}

So Morse functions are dense in the ring $C^\infty(M,\mathbb{R})$,
and a fortiori, Morse functions always exits.

Morse functions give quite a bit of topological information on the manifold,
specially in the compact case. 



One important source of complexes is the Morse-Smale complex,
which arises in a natural way.


\section{Fundamental theorems of Morse theory}

\begin{theorem}[Morse lemma]
\cite{mat1997}
Let $M^n$ be a manifold, and $f$ a Morse function on $M$.
Let $x_0$ be critical point of $f$ of index $k$.


Then there is a neighbourhood $V$ of $x_0$ and a chart 
$y:U\to V\subset M$. so that:

$$
f\circ y (y_1,\ldots,y_n)
=
-y_1^2
-\cdots
-y_k^2
+y_{k+1}^2
+\cdots
+y_n^2
+C
$$

Where $C$ is a constant.
\end{theorem}

Let $f$ be a Morse function on $M$,
We notice that in the compact case $f$ has 
a minimum, $m$ and a maximum $M$
achieved at points $x_m$ and $x_M$.

Then $M^{x_m}=\emptyset$ and $M^{x_M}=M$.
This hints that we can restruct $M$ from how the sets $M^t$
change as $t$ moves through the real line.

In particular, fix some $\varepsilon$ which will be assumed to be
as small as needed, we wonder if there is any relation between $M^t$
and $M^{t+\varepsilon}$. It turns out, as Morse \cite{mor1934} showed
we have that if $f$ does not have any critical point in the interval 
$(t,t+\varepsilon)$ then
 $M^t$ and $M^{t+\varepsilon}$ are difeomorphic. The proof is
 in fact quite elementary.
 
 A more subtle thing Morse proved is what happens when $M^t$
 crosses over a critical point. In this cased he showed that,
 if choose $\varepsilon$ so that $Crit(f)\cap (t,t+\varepsilon)$
 has exactly one point (which we can always do by hypothesis)
 then
  $M^{t+\varepsilon}$ can be obtained by adjoining a $k$-cell to $M^t$
 where $k$ happens to be the index of the unique critical point in 
$(t,t+\varepsilon)$.


In summary

\begin{theorem}
\cite{mor1934}
Let $\varepsilon>0$ so that
$$Crit(f)\cap (t,t+\varepsilon)=\emptyset$$

Then   $M^{t}$ and $M^{t+\varepsilon}$ 
are difeomorphic.


\end{theorem}

\begin{theorem}
\cite{mor1934}
Let $\varepsilon>0$ so that
$$Crit(f)\cap (t,t+\varepsilon)=\{x_0\}$$

Let $k$ be the index of $x_0$. Then

Then  $M^{t+\varepsilon}$ 
is homotopy equivalent to $M^t\cup_\phi (D^{n-k}\times D^k)$ where
$D^k$ is the $k$-disk and $\phi$ is and adjunction map
between $\partial (D^{n-k}\times D^k)$ and $\partial M^t$.


\end{theorem}


\begin{corollary}
\cite{sma1961}
Any compact manifold 
of dimension $n$
is homotopy equivalent to a space such as the following:

$$
D^n\cup
D^{n-k_1}\times D^{k_1}
\cup
\cdots
\cup
D^{n-k_m}\times D^{k_m}
$$
\end{corollary}


\begin{theorem}[Morse inequalities]
Let $M$ be a compact manifold, and $b_i$ it's betti numbers.
Let $f$ be a Morse function on $M$. And let $\lambda_k$ be the 
number of $k$-index critical points of $f$. Then we have that:

$$
\lambda_i\geq b_i,\quad
\forall i\in\mathbb{N}
$$

\end{theorem}

\section{An application of Morse theory}

In this section we describe how to use Morse theory to classify closed $1$-manifolds.
In particular, we show the well-know theorem:

\begin{theorem}
A closed connected $1$-manifold is diffeomorphic to $\mathbb{S}^1$.
\end{theorem}

We briefly mention that it can be used to derive the usual classification of $2$-manifolds.





\section{The Morse-Smale complex}

Morse functions give a natural complex which computes the homology
of the manifold.

We define the chain groups 
to be:

$$
C_f(f)
\defequal
\left\{
\sum
_
{c\in Crit_k(f)}
a c
\vert
a\in \mathbb{Z}_2
\right\}
$$

And we define the differential operator $\partial$ to be:

$$
\partial a
=
\sum
_{b\in Crit_k(f)}
n(a,b)b
$$

Where $n(a,b)$ is the number of trajectories flowing from $a$
to $b$ through the gradient flow. This number is finite, and
we have that $\partial^2=0$.




\bibliographystyle{plain}

\bibliography{tfm}

\end{document}
